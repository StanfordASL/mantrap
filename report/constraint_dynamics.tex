\subsection{Dynamics Constraints}
\label{text:approach/constraint/dynamics}
The dynamics constraints subside all constraints in problem \ref{problem:general} that are not directly related to safety, i.e. 

\begin{align}
\label{eq:constraint_f}
&\x_{t+1} = \f(\x_t, \u_t) \qquad& \forall t \in [0, T - 1] \\
\label{eq:constraint_x}
&\x_t \in \xset & \forall t \in [0, T]\\
\label{eq:constraint_u}
&\u_t \in \uset & \forall t \in [0, T]\\
\label{eq:constraint_x0}
&\x_0 \in \xset_0
\end{align}

Using the shooting trajectory optimization paradigm, the decision variables are the robot's control inputs $\u_t \in \uset$. The states $\x_t$ are obtained by unrolling the controls starting at some initial state $\x_0 \in \xset_0$ and iteratively using the robot's dynamics $\f(\cdot)$. \ac{IPOPT} on the other side solves a general \ac{NLP} of the form \cite{Wachter2006} \\

\begin{problem}{General IPOPT problem formulation}
\begin{align}
\min_{x \in \mathbb{R}^n} \quad & \f(x) \\
\textrm{s.t. } \quad & g^L \leq g(x) \leq g^U \\
& x^L \leq x \leq x^U 
\end{align}
\label{problem:general_ipopt}
\end{problem}

As indicated in Section \ref{text:approach/formulation}, the bounds of the controls $\u_t$ have a shape, as the decision variable $\x$ in problem \ref{problem:general_ipopt}. Therefore \ref{eq:constraint_u} is implicitly posed, without further ado, just by using the robot's control input bounds as bounds of the decision variable. Also, the constraints \ref{eq:constraint_f} and  \ref{eq:constraint_x0} are satisfied by the problem design, using the shooting method, leaving merely the constraint \ref{eq:constraint_x} to be explicitly defined. The state $\x$ of the robot incorporates both its position and its velocity. While the prediction models, objectives, and constraints only depend on relative measures regarding the agent's positions, the positional subset of $\xset$ can be safely assumed to be unbounded ($\xset_{pos} = \mathbb{R}^2$). Due to the speed boundaries imposed on the robot ($||v||_1 \leq v_{max}$, comp. Section \ref{text:approach/formulation}), in order for the solution to be feasible, a maximal speed constraint must be established:

\begin{equation}
\x_t \in \xset \Rightarrow -v_{max} \leq g_{v_{max}}(\x_t) = \dot{\x}_t \leq v_{max} \quad \forall t \in [0, T]
\label{eq:constraint_v_max}
\end{equation}

Computing the Jacobian for constraint \ref{eq:constraint_v_max} is straightforward using the chain rule and exploiting the linear robot dynamics (as above when deriving the goal objective's gradient in equation \ref{eq:goal_gradient_dynamics}):  

\begin{align}
\nabla g_{v_{max}} &= \pd{g_{v_{max}}}{\u_{0:T-1}} = \pd{g_{v_{max}}}{\x_{0:T}} \cdot \pd{\x_{0:T}}{\u_{0:T-1}} \\
\Rightarrow \pd{g_{v_{max}}}{\x_{0:T}} &= \begin{bmatrix} \pd{g_{v_{max}}^1}{\x_{0:T}} & \hdots & \pd{g_{v_{max}}^T}{\x_{0:T}}\end{bmatrix}^T  \\
&= \begin{bmatrix} 
0 & 0 & 1 & 0 & 0 & 0 & 0 & 0 & \hdots & 0 \\ 
0 & 0 & 0 & 1 & 0 & 0 & 0 & 0 & \hdots & 0 \\  
0 & 0 & 0 & 0 & 0 & 0 & 1 & 0 & \hdots & 0 \\
0 & 0 & 0 & 0 & 0 & 0 & 0 & 1 & \hdots & 0 \\ 
\vdots & \vdots & \vdots & \vdots & \vdots & \vdots & \vdots & \vdots & \vdots & \vdots \\
0 & 0 & 0 & 0 & 0 & 0 & 0 & 0 & \hdots & 1 \end{bmatrix} \\
\Rightarrow \pd{\x_{0:T}}{\u_{0:T-1}} &= \begin{bmatrix} \mathbf{0}_{n \times m} \\ B_n \end{bmatrix}
\end{align}

