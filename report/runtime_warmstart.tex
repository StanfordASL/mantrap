\subsection{Warm Starting}
\label{text:approach/runtime/warm_starting}
It is widely known that warm starting an optimization can be very beneficial for its convergence speed for several optimization algorithms, e.g. shown in \cite{Banerjee2020} for \ac{GuSTO} or for \ac{IPOPT} in \cite{Shahzad2010}, \cite{John2008} or \cite{Spielberge2019}.
\newline
Within project \project the algorithm has been warm-started by solving the optimization problem posed in problem \ref{eq:formulation} while ignoring all objectives and constraints that relate to pedestrians, i.e. by solving the following optimization problem: \\

\begin{problem}{Simplified \project optimization problem for warm-starting}
\begin{align}
\min_{\u_{0:T-1}} \quad & J_{goal}(\x_{0:T}) \\
\textrm{s.t. } \quad & \x_{t+1} = \f(\x_t, \u_t) & \forall t \in [0, T - 1] \\
& \x \in \xset & \forall t \in [0, T]\\
& \u_t \in \uset & \forall t \in [0, T]\\
& \x_0 \in \xset_0
\end{align} 
\label{eq:formulation_warm_starting}
\end{problem}

Since the simplified optimization problem in problem \ref{eq:formulation_warm_starting} does not depend on the pedestrian dynamics model $\tilde{\Phi}$ it is much easier and more efficient to solve, being a convex quadratic program. \\

\begin{figure}[!ht]
\begin{center}
\begin{tikzpicture}

    \node (R) at (0, 0) [circle, shade, draw] {Robot};
    \node (G) at (8, 4) [circle, shade, draw] {Goal};
    \node (P1) at (2, 3) [circle, fill=orange, draw] {$P_1$};
    \node (P2) at (8, 1) [circle, fill=yellow, draw] {$P_2$};
    
    \tkzDefPointBy[projection=onto R--G](P1)  \tkzGetPoint{P1m}
    \tkzDefPointBy[projection=onto R--G](P2)  \tkzGetPoint{P2m}
    
    \draw[->, dotted, very thick] (R) to node[above] {} (G);
    \draw (P1m) -- (P1) node[midway, sloped, above] {$\mu_{P1}$};
    \draw (R) -- (P1m) node[midway, sloped, above] {$\eta_{P1}$};
    \draw (P2m) -- (P2) node[midway, sloped, above] {$\mu_{P2}$};
    \draw (R) -- (P2m) node[midway, sloped, below] {$\eta_{P1}$};
    
\end{tikzpicture}
\end{center}
\label{img:robot_goal_encoding}
\end{figure}

%Does the pre-computed set contain every possible scenario? No, but just warm-start so not required, there will be a constrained optimization afterward anyway
