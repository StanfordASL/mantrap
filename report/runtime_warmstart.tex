\subsection{Warm Starting}
\label{text:approach/runtime/warm_starting}
It is widely known that warm starting an optimization can be very beneficial for its convergence speed for several optimization algorithms, e.g. shown in \cite{Banerjee2020} for \ac{GuSTO} or for \ac{IPOPT} in \cite{Spielberg2019}. John et alt. The difficulty of warm-starting interior-point methods by exploiting the solution of the preceding problem, has already been stated in \cite{Wright1997}\cite{Gondzio2008}. Mainly, this difficulty originates from the problem that the solution of the preceding problem is close to the feasible boundary, which ultimately leads to decreasing step lengths and thus sticking the search to a small region around the previous solution. As a consequence, the solver is stuck to the solution of the initial problem, which might be (quite) sub-optimal or even infeasible. \cite{John2008} present several strategies for warm-starting interior-point methods for linear programs specifically, similarly \cite{Shahzad2010}\cite{Gondzio2008}. However, all of these works are restricted to specific kinds of problems, such as linear or quadratic programs. As none of these can be safely assumed for Problem \ref{problem:general}, the warm-starting method used within this project, tries to find a good approximative solution independent from the preceding iterations and individually contrived for the specific problem of crowd navigation.
\newline
Within project \project, the algorithm has been warm-started by solving the optimization problem posed in problem \ref{problem:general} while ignoring all objectives and constraints that relate to pedestrians, i.e., by solving the following optimization problem: \\

\begin{problem}{Simplified optimization problem for warm-starting}
\begin{align}
\min_{\u_{0:T-1}} \quad & J_{goal}(\x_{0:T}) \\
\textrm{s.t. } \quad & \x_{t+1} = \f(\x_t, \u_t) & \forall t \in [0, T - 1] \\
& \x \in \xset & \forall t \in [0, T]\\
& \u_t \in \uset & \forall t \in [0, T]\\
& \x_0 \in \xset_0
\end{align}
\label{problem:general_warm_starting}
\end{problem}

Since the simplified optimization problem in problem \ref{problem:general_warm_starting} does not depend on the pedestrian dynamics model $\tilde{\Phi}$, it is easier and more efficient to solve, being a convex quadratic program. Next to the presented way of warm-starting the trajectory optimization algorithm, there are several other ways. Section \ref{text:experiments/unit} compares this approach to several other warm-starting algorithms, such as by leveraging pre-computation (similar to \cite{Merkt2018}).