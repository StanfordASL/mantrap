
\section{Problem Formulation}
\label{text:approach/formulation}
In this work, we are interested in finding the optimal trajectory for a robot navigating among pedestrians on the two-dimensional plane. Let $\x_t = (x_t, y_t) \in \mathbb{R}^2$ and $\dx_t = (\dot{x}_t, \dot{y}_t) \in \mathbb{R}^2 $ be the position and velocity of the robot, also referred to as $ego$, at time $t$ and $(\x, \dx)_{0:N}$, a trajectory over multiple position-velocity-pairs $((\x, \dx)_0, (\x, \dx)_1, (\x, \dx)_2, \hdots, (\x, \dx)_N)$. Within this work, the robot is assumed to have double integrator dynamics. With control input $\u_t = (u_{x, t}, u_{y, t}) \in \mathbb{R}^2$, we have: 

\begin{equation}
\ddx_t = \u_t
\label{eq:robot_dynamics}
\end{equation}

For further simplification, the robot's dynamics are assumed to be deterministic. 
The pedestrians (also referred to as $ados$) follow single integrator dynamics. As pointed out in \cite{Ivanovic2018}, this is a natural choice "as a person's movements are all position-changing, e.g., walking increases position along a direction, running does so faster." Other standard models for pedestrian dynamics such as Social Forces \cite{Helbing1995}, however, regard the pedestrian to be a double integrator, not a single integrator, and describe the forces acting on it introduced by other pedestrians and obstacles. Having a reasonable fast reaction time and a large maximal acceleration in comparison to the robot, both of these descriptions converge so that the single integrator model is the right choice nonetheless. \footnote{As discussed in Chapter \ref{text:experiments} the modular implementation allows us to use different pedestrian dynamics. When testing against other prediction environments non-single integrator dynamics are used, but most analysis described in this report relates to the single integrator model.}

\begin{align}
\dxped[k]_t = \uped[k]_t
\label{eq:pedestrian_dynamics}
\end{align}

Each pedestrian's future trajectory is predicted as a probabilistic and multimodal using some model $\distmodel[]$. Thereby, let $\xped[k]_t \sim \dist[k]_t$ be the distribution of the pedestrian $k$s velocity at time $t$, with mean $\muped[k]_t$, variance $\sigmaped[k]_t$, and mode-weights vector $\piped[k]_t$. The prediction is based on the past states of the robot $\x_{0:t}$ and of every pedestrian in the scene $\xped[j]_{0:t}$, including pedestrian $k$. Moreover, the function $\distmodel[]$ is generally not shared across all pedestrians but individually for each one.

\begin{align}
\xped[k]_t &\sim \distmodel[k]_t(\muped[k]_t, \sigmaped[k]_t, \piped[k]_t) \\
\dist[k]_t &= \distmodel[k] (\x_{0:t}, \xped[0]_{0:t}, \hdots, \xped[K]_{0:t})
\end{align}

Like the robot, the pedestrians are modeled as single point masses, both underlying speed bounds defined by the $L_2$ norm of their velocities. Furthermore, both the robot and the pedestrians underly constraints for their minimal and maximal control effort. Thus, the sets of feasible control inputs $
\uset$ and $\upedset$ respectively, are defined using the $L_1$-norm and the $L_2$-norm respectively:

\begin{align}
\uset &= \{\u | \u \in \mathbb{R}^2, ||\u||_1 \leq u_{max}\} \\
\upedset &= \{\uped[] | \uped[] \in \mathbb{R}^2, ||\uped[]||_2 \leq \tilde{u}_{max}\} 
\label{eq:controls_bounds}
\end{align}
 
All actions are assumed to take place in a free-space, two-dimensional environment. 
\newline
Within project \project, we want to find a robot trajectory $\x_{0:T}$ over some discrete-time horizon $N$ that makes trade-offs between minimizing the travel time from its current state to some goal state in $\xset_f = \{\boldsymbol{g} | \boldsymbol{g} \in \xset \}$ on the one side and the interference with the pedestrians in the scene, concerning its dynamic as well as safety boundaries, on the other side. For further simplification, perfect knowledge about the current and all past states of surrounding agents $\xped[k]_{0:t} \forall k \in [0, K]$ is assumed.
\newline\newline
For brevity of notation, in the following, the temporal index $t$ and the pedestrian index $k$ will be omitted when not necessary.

