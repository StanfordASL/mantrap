\chapter{Appendix - Math Prerequisites}
\label{appendix:gmm_log_prob}
For a two-dimensional \ac{GMM} $g \sim GMM(\mu, \Sigma, \pi, \rho)$ with mean vector $\mu$, variance matrix $\Sigma$, mode importance vector $\pi$ and Pearson's correlation coefficients $\rho$ the probability at $(x, y)$ can be computed as:\footnote{\href{https://de.wikipedia.org/wiki/Mehrdimensionale_Normalverteilung}{Wikipedia - Multivariate Normal Distribution}}.

\begin{align}
f_m(x, y | \mu, \sigma, \rho) 
&= \frac{1}{2 \pi \sqrt{\det \Sigma}} \exp \left(- \frac{1}{2} (\boldsymbol{x} - \boldsymbol{\mu})^T \Sigma (\boldsymbol{x} - \boldsymbol{\mu}) \right) \\
&= {\frac{\sqrt{1- \rho^2}}{2 \pi \sigma_x \sigma_y} 
\exp - \frac{1}{2 (1 - \rho^2)}} \left( \frac{( x - \mu _x)^2}{\sigma_x^2} +
\frac {(y - \mu_y)^2}{\sigma_y^2}-{\frac {2\rho (x - \mu_x)(y - \mu_y)}
{\sigma_x \sigma_y}} \right)	
\end{align}

The Pearson coefficient is the covariance for two normalized random variables $X$ and $Y$, in their normalized form $\tilde{X} = (X - \mu_x)/\sigma_x$ and $\tilde{Y} = (Y - \mu_y) / \sigma_y$. Using the law of linear transformations of covariances we get:\footnote{\href{https://en.wikipedia.org/wiki/Pearson_correlation_coefficient}{Wikipedia - Pearson correlation coefficient}} 

\begin{equation}
Cov(\tilde{X}, \tilde{Y}) = \frac{1}{\sigma_x \sigma_y} Cov(X, Y) = \rho_{X, Y}
\end{equation}

The expected value over all modes $m \in [0, M]$ super-imposes the \ac{PDF} for the bi-variate Gaussian distributions, weighting each one by its importance $\pi_m$:

\begin{equation}
\mathbb{E}_{GMM}[x, y] = \sum_{m=0}^M \pi_m \cdot f_m(x, y) =  \sum_{m=0}^M \exp \left( \log \pi_m + \log f_m(x, y) \right)	
\end{equation}


\chapter{Appendix - Detailed Planning Approach Comparison}
There are several categories from which the properties of planning approaches can be evaluated: 

\begin{itemize}
    \item Optimality: trajectory cost vs minimal possible (optimal) cost
    \begin{itemize}
        \item globally: $J(x(t), u(t)) = J^*(x(t), u(t)) \forall t$
        \item locally: $J(x(t), u(t)) \approx J^*(x(t), u(t)) \forall x(t) + \epsilon, u(t) + \epsilon $
        \item not at all
    \end{itemize}
    
    \item Risk-Awareness: Point-wise constraints are easier to fulfill since each state can be regarded individually, however it assumes independence between the states, which is clearly not given due to the dynamical constraints of the ego. As shown in \cite{JansonSP15} both the additive and multiplicative formulation do not scale with increasing planning horizon, since the accumulated risk converges to infinity. Also the using point-wise constraint often is a very conservative choice, as it does not allow to take more risk at some point to be more efficient or to save risk somewhere else. 
    \begin{itemize}
        \item trajectory-wise: $\sum_{k = 0}^N r(x_k, u_k) \leq R_{max}$
        \item point-wise: $r(x_k, u_k) \leq R_{max} \forall k$
    \end{itemize}

    \item Computational Feasibility:  
    \begin{itemize}
        \item real-time planning
        \item online policy/trajectory correction (e.g. using perturbation) 
        \item not real-time applicable
    \end{itemize}
    
    \item Explainability/Tractability \& Guarantees: When the result of the planning algorithm is not tractable it is very hard (or even impossible) to give theoretical guarantees with respect to safety constraints. Therefore for these approaches an additional step, determining the empirical risk of collision using Monte Carlo simulations, is necessary, at the cost of additional run-time.  
\end{itemize}

\begin{center}
\begin{tabular}{c||p{2cm}|p{1cm}|p{2cm}|p{1cm}|p{1cm}|p{1cm}|p{1cm}|p{1cm}}
    Method & 
    \rotatebox[origin=c]{90}{Space ?} &  \rotatebox[origin=c]{90}{Risk-Awareness ?} & 
    \rotatebox[origin=c]{90}{PPDF Model ?} & 
    \rotatebox[origin=c]{90}{Risk as cost ?} &
    \rotatebox[origin=c]{90}{Optimality ?} &
    \rotatebox[origin=c]{90}{Parametrization ?} &
    \rotatebox[origin=c]{90}{Interactive ?} &
    \rotatebox[origin=c]{90}{Tractability ?} \\
    \hline\hline
    
    CCMPC & continuous & point-wise & approximate (uni-Gaussians) & no & locally & sparse & no & yes \\
    \hline
    SIPP & discrete & point-wise & accurate (sampled GMM) & no & globally & sparse & no & yes \\
    \hline
    DRL & continuous & none & none & no & none & much & yes & no \\
    \hline
    PORCA & continuous & none & none & none & none & much & yes & no \\
    \hline
    SACBP & continuous & yes & accurate (sampled traject.) & yes & locally & sparse & no & yes \\
    \hline
    POMDP & discrete & yes & accurate & yes & globally & much (state definition) & no & yes \\
    \hline
    IRL & discrete & yes & accurate & none & none & much (optimal trajectory) & medium & medium
    
\end{tabular}
\end{center}
%